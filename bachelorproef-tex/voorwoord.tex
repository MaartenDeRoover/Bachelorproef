%%=============================================================================
%% Voorwoord
%%=============================================================================

\chapter*{Woord vooraf}
\label{ch:voorwoord}
\newglossaryentry{KMO}{
	type=\acronymtype,
    name={KMO},
    description={Kleine of Middelgrote Onderneming},
    plural={KMO's}
}
\newglossaryentry{SEO}{
	type=\acronymtype,
    name={SEO},
    description={Search Engine Optimization},
    first={Search Engine Optimization (SEO) \glsadd{SEOG}},
    see=[Glossary:]{SEOG}
}
\newglossaryentry{SEOG}{
	name={SEO},
    description={Search Engine Optimization of zoekmachineoptimalisatie is het optimaliseren van de kans dat een website hoog scoort in de organische zoekresultaten op een zoekmachine}
}
\newglossaryentry{SEA}{
	type=\acronymtype,
    name={SEA},
    description={Search Engine Advertising},
    first={Search Engine Advertising (SEA)\glsadd{SEAG}},
    see=[Glossary:]{SEAG}
}
\newglossaryentry{SEAG}{
	name={SEA},
    description={Search Engine Advertising of zoekmachinemarketing is het tegen betaling laten verschijnen van advertenties voor en naast de zoekmachineresultaten}
}

Deze bachelorproef is geschreven in opdracht van Dropsolid, het bedrijf waar ik stage en vakantiejob heb gedaan. Dropsolid is een \gls{KMO} gelegen te Gent die in 2012 werd opgericht en ondertussen al meer dan 50 werknemers telt. De missie van Dropsolid is om digitaal ondernemen voor bedrijven makkelijker te maken. Dit doen ze door de klant tijdens elke fase van hun digitale transformatie bij te staan. Concreet wil dit zeggen dat ze een digitaal platform voor de klant opzetten met behulp van Drupal en hen trainen in hoe ze dit digitaal platform kunnen gebruiken. Ook ondersteunen ze klanten door hen advies te geven over digitale marketing, \gls{SEO} en \gls{SEA}.

Het onderwerp van deze bachelorproef werd mij aangereikt tijdens mijn stage door Nick Veenhof, de CTO van Dropsolid. Tijdens mijn stage bespraken we enkele onderzoeken die interessant voor Dropsolid zouden zijn. Ik koos voor dit onderwerp omdat dit mij het interessantste leek en het dichtst tegen mijn sterkten aanleunde.

Graag zou ik Rembrand Le Compte, Technical Lead bij Dropsolid, en mijn co-promotor Nick Veenhof bedanken om mij in de juiste richting te sturen tijdens mijn onderzoek. Verder zou ik graag alle developers van Dropsolid bedanken omdat ik steeds bij hen terecht kon voor vragen.

Ook zou ik graag mijn promotor Steven Vermeulen bedanken voor het vertrouwen en de hulp tijdens het onderzoeksproces. Steven was altijd beschikbaar en antwoordde snel op al mijn vragen.
%% TODO:
%% Het voorwoord is het enige deel van de bachelorproef waar je vanuit je
%% eigen standpunt (``ik-vorm'') mag schrijven. Je kan hier bv. motiveren
%% waarom jij het onderwerp wil bespreken.
%% Vergeet ook niet te bedanken wie je geholpen/gesteund/... heeft

